\newglossaryentry{glossar.projectional}
{
  name=Projektionseditor,
  description={Editiert Programmcode direkt auf der Ebene des
               abstrakten Syntaxbaums eines Programms}
}

\newglossaryentry{glossar.modell}
{
  name=Modell,
  description={Reduziert ein komplexes Universum auf eine überschaubare Welt}
}

\newglossaryentry{glossar.metamodell}
{
  name=Metamodell,
  description={Ein Modell über Modelle}
}

\newglossaryentry{glossar.abstraktersyntaxbaum}
{
  name=Abstrakter Syntaxbaum,
  description={Datenstruktur die Programmcode auf abstrakte Weise repräsentiert}
}

\newglossaryentry{glossar.semiotik}
{
  name=Semiotik,
  description={Wissenschaft die sich mit den Gesetzmäßigkeiten von Sprache beschäftigt}
}

\newglossaryentry{glossar.taxonomie}
{
  name=Taxonomie,
  description={„Glossar mit hierarchischer Einordnung der Begriffe in Kategorien“ \citep[S. 3-4]{Drachenfels}}
}

\newglossaryentry{glossar.glossar}
{
  name=Glossar,
  description={„Liste von Begriffen mit Erklärungen“ \citep[S. 3-4]{Drachenfels}}
}

\newglossaryentry{glossar.thesaurus}
{
  name=Thesaurus,
  description={„Taxonomie mit zusätzlichen Synonym-/Ähnlichkeits-Beziehungen“ \citep[S. 3-4]{Drachenfels}}
}

\newglossaryentry{glossar.onologie}
{
  name=Ontologie,
  description={„Glossar mit Axiomen und beliebigen Relationen“ \citep[S. 3-4]{Drachenfels}}
}

\newglossaryentry{glossar.axiome}
{
  name=Axiom,
  description={„Axiome sind Aussagen, die in der Domäne immer wahr sind“ \citep[S. 3-4]{Drachenfels}}
}


\newglossaryentry{glossar.verw}
{
  name=Reichhaltige Verweisungen,
  description={Gehen über normale Verweisungen hinaus. Sie sind ausführbarer Programmcode und greifen auf (u.U. berechnete) Attribute eines Dokumentelements zu}
}

\newglossaryentry{glossar.dokelem}
{
  name=Dokumentelement,
  description={Grundlegende Bestandteile des inneren Aufbaus von Dokumenten.
  Beispiele hierfür sind Kapitel, Absätze oder Abbildungen etc}
}

\newglossaryentry{glossar.aktor}
{
  name=Aktor,
  description={Entitäten mit nebenläufiger Rechenfähigkeit die über Nachrichtentausch kommunizieren}
}

\newglossaryentry{glossar.domaene}
{
  name=Domäne,
  description={Bezeichnung eines Fachgebietes bzw. Wissensgebietes}
}

\newglossaryentry{glossar.domaeneeditor}
{
  name=Domäneneditor,
  description={Ein grafischer Editor der es ermöglicht ein Modell eines spezifischen Fachgebiets zu manipulieren}
}

\newglossaryentry{glossar.graph}
{
  name=Graph,
  description={Eine Struktur die aus einer Menge von Knoten und Kanten besteht. Die Graphentheorie ist ein Gebiet der diskreten Mathematik}
}

\newglossaryentry{glossar.baum}
{
  name=Baum,
  description={Ein zusammenhängender kreisfreier Graph}
}

\newglossaryentry{glossar.konsistenz}
{
  name=Konsistenz,
  description={Ein (wissenschaftliches) Dokument ist dann konsistent, wenn seine Inhalte insich stimmig bzw. widerspruchsfrei sind. Wenn ein Text Bezug zu einem Detail eines Domänenmodell nimmt, sich dieses Detail jedoch ändert, aber im Text noch immer zum ursprünglichen Detail verwiesen wird, dann hat das Dokument eine Inkonsistenz}
}

\newglossaryentry{glossar.templ}
{
  name=Template,
  description={Eine programmierbare Vorlage die mit statischen und dynamischen Bestandteilen ausgestattet ist. Die dymanischen Inhale werden aus einer Datenstruktur eines Programms bezogen und sind daher variabel}
}

% Acronyme
% --------

\newacronym{glossar.dsl}{DSL}{Domain Specific Language. Eine Programmiersprache die auf eine
spezifische Anwednungsdomäne zugeschnitten ist}

\newacronym{glossar.ast}{AST}{Abstract Syntaxtree bzw. abstrakter Syntaxbaum}

\newacronym{glossar.svg}{SVG}{Scalable Vector Graphics. Ein standardisiertes Bildformat für Vektorgrafiken}
