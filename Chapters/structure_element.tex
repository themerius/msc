  \begin{itemize}
  \itemsep1pt\parskip0pt\parsep0pt
  \item
    \textbf{half title}, Schmutztitel, Kurzsachtitel, Vortitel
  \item
    \emph{title}

    \begin{itemize}
    \itemsep1pt\parskip0pt\parsep0pt
    \item
      \textbf{document titel}, Sachtitel
    \item
      \textbf{outline title}

      \begin{itemize}
      \itemsep1pt\parskip0pt\parsep0pt
      \item
        \textbf{chapter title}, Kopftitel {[}Hiller{]}
      \end{itemize}
    \end{itemize}
  \item
    \emph{subtitle}

    \begin{itemize}
    \itemsep1pt\parskip0pt\parsep0pt
    \item
      \textbf{document subtitle}, Untertitel
    \item
      \textbf{outline subtitle}
    \end{itemize}
  \item
    \textbf{author}, Autor
  \item
    \textbf{outline}

    \begin{itemize}
    \itemsep1pt\parskip0pt\parsep0pt
    \item
      \emph{part}
    \item
      \emph{chapter} (Kurze einleitende Inhaltsangabe eines Buches oder
      Buchabschnittes -\textgreater{} oft nummeriert und in groesserer
      fetter Schrift {[}Hiller S. 176{]})
    \item
      \emph{section}, Abschnitt
    \item
      \textbf{subsection}
    \item
      \textbf{subsubsection}
    \end{itemize}
  \item
    \emph{paragraph}, Absatz
  \item
    \emph{list}, Aufzaehlung
  \item
    \textbf{quotation}, Zitat

    \begin{itemize}
    \itemsep1pt\parskip0pt\parsep0pt
    \item
      \emph{complex run-in quotation}
    \item
      \emph{block quotation}
    \end{itemize}
  \item
    \textbf{numbering}, Benummerung

    \begin{itemize}
    \itemsep1pt\parskip0pt\parsep0pt
    \item
      \textbf{outline numbering}
    \item
      \emph{figure numbering}
    \item
      \emph{table numbering}
    \item
      \textbf{formular numbering}, Formelzaehler
    \end{itemize}
  \item
    \emph{sentence}
  \item
    \emph{caption} same\_as \emph{legend}, (Bild-)Legende (Text der
    einem anderen Item beigefügt wird; Informierender Text der explizit
    ein anderes Item erklärt ==\textgreater{} In DoCO war legend von
    caption abgeleitet, aber keine Ahnung was da der Unterschied sein
    soll)
  \item
    \emph{captioned box}, \textbf{flow block environment} {[}hat ggf.:
    numbering; legend, title/subtitle{]} (Block, Inline Elemente?)

    \begin{itemize}
    \itemsep1pt\parskip0pt\parsep0pt
    \item
      \emph{table}, Tabelle
    \item
      \emph{figure}, Bild
    \item
      \emph{formula}, Formel {[}hat: Formelzähler{]} (Kann auch Inline
      vorkommen!)

      \begin{itemize}
      \itemsep1pt\parskip0pt\parsep0pt
      \item
        \textbf{math}
      \item
        \textbf{physical}
      \item
        \textbf{chemistry}
      \end{itemize}
    \item
      \textbf{code} (Kann auch Inline vorkommen!)
    \end{itemize}
  \item
    \textbf{annotation}?, \textbf{note}?, Anmerkung

    \begin{itemize}
    \itemsep1pt\parskip0pt\parsep0pt
    \item
      \emph{footnote}, Fussnote
    \item
      \textbf{marginal note}, Marginalie
    \item
      \textbf{ruby annotation}, (s.
      \url{http://de.wikipedia.org/wiki/Ruby_Annotation} gefunden in
      ISO)
    \item
      \emph{text box}?, \textbf{digression}, Exkurs
    \item
      \emph{refernece}, Verweisung (Referenzierung zu speziellen Teil
      des Dokuments oder zu anderen Dokumenten)

      \begin{itemize}
      \itemsep1pt\parskip0pt\parsep0pt
      \item
        \emph{label} (Identifizieren eines Elements im Dokument, z.B.
        ``Abb. 1'' innerh. des Textes)

        \begin{itemize}
        \itemsep1pt\parskip0pt\parsep0pt
        \item
          \emph{outline label}
        \item
          \emph{figure label}
        \item
          \emph{table label}
        \item
          \textbf{formular label}
        \item
          \textbf{code label}
        \end{itemize}
      \item
        \emph{bibliographic reference}
      \end{itemize}
    \end{itemize}
  \item
    \textbf{emphasis}?, \textbf{mark up}?, Auszeichnung

    \begin{itemize}
    \itemsep1pt\parskip0pt\parsep0pt
    \item
      \textbf{italic}, Kursiv
    \item
      \textbf{semibold}, Halbfett
    \item
      \textbf{small caps}, Kapitaelchen
    \item
      \textbf{capital letters}, Versalien
    \item
      \textbf{other font}, Andere Schrift
    \item
      \textbf{indented paragraph}, Eingerueckter Absatz
    \end{itemize}
  \end{itemize}