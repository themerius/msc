Naturwissenschaftliche Dokumentation hat mich persönlich schon immer fasziniert.
Wie sich jedoch herausstellt, steckt hinter den Textwerkzeugen wie z.B. Microsoft Word
nicht sonderlich viel „Magie“. Nahezu alles muss immer noch von Hand eingetippt
bzw. Schaubilder eingefügt werden -- das ist für meinen Geschmack noch sehr manuell und fehlerträchtig.

Daher wollte ich mich in meiner Bachelorarbeit \citep{Hodapp} mit diesem Thema
richtig auseinandersetzen.
Das die Evolution solcher Werkzeuge noch nicht abgeschlossen ist, wurde
mir, am Fraunhofer-Institut für Solare Energiesysteme ISE während meines Praxissemsters, klar:
es herrscht Unzufriedenheit mit den momentanen Werkzeugen zur Generierung des Jahresberichts der Abteilung.
In meiner Freizeit habe ich Ideen gesammelt und diese hielten sich hartneckig
bis zur Abschlussarbeit.
Am Fraunhofer-Institut für Algorithmen und Wissenschaftliches Rechnen SCAI stieß ich auf
offene Ohren; Marc Zimmermann gemeinsam mit Makro Boger betreuen meine Abschlussarbeit.

Im Anschluss folgte das Masterstudium. Die Bachelorarbeit machte klar,
dass es noch Potential gibt -- ich wollte daran weiterarbeiten.
Ich war also auf der Suche nach neuen Impulsen und Ideen, um daraus eine Masterarbeit zu machen.
Prof. Boger zeigte mir Markus Völters \emph{mbeddr}, eine IDE mit DSLs für eingebettete Systeme.
Mbeddr kann auch LaTeX-Dokumentation generieren; ähnlich der Methodik meiner Arbeit.
Der Clou ist jedoch, dass die IDE ein \emph{projectional editor} ist --
der fehlende Impuls! Im Master Seminar durfte ich mich damit auseinandersetzen,
was mich ermuntert hat, projectional editing für meine Masterarbeit aufzugreifen.