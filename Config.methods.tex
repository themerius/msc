\setboolean{draft}{true}

\newcommand{\autor}{Sven Hodapp}
\newcommand{\documenttype}{Masterarbeit}
\newcommand{\thema}{Wissenschaftliche Dokumente mit domänenspezifischen Inhalten:
Entwicklung eines allgemeinen Modells für wissenschaftliche Dokumente umgesetzt als browserbasierter Editor}
\newcommand{\subtitle}{
  Lorem ipsum dolor sit amet, consetetur sadipscing elitr
}
\newcommand{\institute}{
  \begin{tabular}{r l}
    Institution: & Hochschule Konstanz \\
    Bereich: & Fakultät Informatik \\
    Betreuer: & Prof. Dr. John Doe \\
    Version: & \today
  \end{tabular}
}

% Schlagworte: http://xgnd.bsz-bw.de
\newcommand{\keywords}{
Wissenschaftliches Manuskript,
Textverarbeitung,
Modelltheorie,
Taxnomomie,
%Metamodell,
%Wissenschaftlich-technische Software,
Software Engineering,
%Rich Internet Application,
%Wissenschaftliche Literatur,
%Dokumentation,
%Druckwerk,
%Gute wissenschaftliche Praxis,
Projectional Editing,
Domain-Specific Languages,
Domain Modelling
}

% Maximal 1200 Anschläge, alles in einem Abschnitt. (Nach DIN 1422-1).
\renewcommand{\abstract}{
Aktuelle Textverarbeitungsprogramme haben gerade bei der Erstellung von
wissenschaftlichen Dokumenten große Defizite, da dort oft formale Modelle,
wie z.B. Chemie-Moleküle, aus dem jeweiligen Fachgebiet (Domäne) benutzt werden müssen.
Heutige Systeme können mit diesen Modellen nicht direkt interagieren,
was oft zu Inkonsistenzen im Dokument führt.
In dieser Arbeit wird der Prototyp eines Dokumenteditors entwickelt,
welcher domänenspezifische Dokumentelemente einführt,
und so formale Modelle direkt bei der Bearbeitung des Dokuments verwendbar macht.
Die Dokumentelemente (z.B. Absätze, Abbildungen etc.) können auch domänenspezifische Inhalte
repräsentieren, z.B. ein Molekül.
Wenn ein Dokumentelement auf ein Anderes verweist,
kann direkt auf Attribute des darunterliegenden Modells zugegriffen werden;
bspw. auf den Namen eines Moleküls.
Der Prototyp beweist seine Praxistauglichkeit anhand von vier konkreten Anwendungsfällen.
Aus dem Prototyp ist eine allgemeine Theorie über den inneren Aufbau von Dokumenten entwickelt
worden und tangiert die Bereiche (Meta) Modellierung, Projektionseditoren, Semiotik und Taxonomie.
Es gibt vielfältige Weiterentwicklungsmöglichkeiten, wie Ausbau zur Open Access Platform mit
erweiterten Autorenfunktionen, Zentrale für Format-Transformationen oder Literate Programming IDE.
}



\newcommand{\abstractEngl}{
Current word processors have deficits for creating academic writings,
because they use often formal models like chemical molecules of a specific domain.
But current systems don't interact with these models, which often leads to inconsistent documents.
In this work a document editor prototype was developed,
which introduces domain specific document elements to enable direct
usage of such models within the document.
Document elements (e.g. paragraphs, figures, etc.) can represent
domain specific content, such as a molecule.
If one document element refers to another one,
it can directly access model attributes of the other one, like the name of the molecule.
The prototype proofes it's practial benefit by four different use cases.
With the help of the prototype it was possible to retrieve general theories
about the inner construction of documents, it covers the areas:
(meta) modelling, projectional editing, semiotics and taxonomy.
It is possible to envolve this thoughts to build a open access platfom
with extended authoring abilities, central for format transformations or
literate programming IDE.
}

\newcommand{\intentionallyBlankPage}{\clearpage\mbox{}\clearpage}

\newcommand{\ausgabedatum}{1. März 2014}
\newcommand{\abgabedatum}{\today}
\newcommand{\autorStrasse}{Hohentwielstraße 2}
\newcommand{\autorPLZ}{78247}
\newcommand{\autorOrt}{Hilzingen}
\newcommand{\autorGeburtsort}{Singen am Hohentwiel}
\newcommand{\autorGeburtsdatum}{16. September 1987}
\newcommand{\prueferA}{Prof. Dr. Marko Boger}
\newcommand{\prueferB}{Dr. Marc Zimmermann}
\newcommand{\firma}{Fraunhofer-Institut für Algorithmen und Wissenschaftliches Rechnen SCAI}
\newcommand{\studiengang}{Master of Science Informatik}

\newcommand{\typesetTitle}{

% Cover

  \begin{titlepage}

  \vspace*{-3.5cm}

  \begin{flushleft}
  \hspace*{-1cm} \includegraphics[width=15.7cm]{Titlepages/htwg-logo}
  \end{flushleft}

  \vspace{2.5cm}

  \begin{center}
    \huge{
      \textbf{\thema} \\[4cm]
    }
    \Large{
      \textbf{\autor}} \\[4cm]
    \large{
      \textbf{Konstanz, \abgabedatum \ifdraft ~(ENTWURF) \fi} \\[2cm]
    }
    
    \Huge{
      \textbf{{\sf MASTERARBEIT}}
    }
  \end{center}

  \end{titlepage}

\setcounter{page}{2}
\intentionallyBlankPage

% Title

  \thispagestyle{empty}
  \setlength{\parskip}{0.5cm}
          \begin{center}
          \textbf{\huge MASTERARBEIT}

          \textbf{zur Erlangung des akademischen Grades}

          \textbf{\Large Master of Science (M. Sc.)}

          \textbf{an der}

          \textsf{\huge Hochschule Konstanz}\\
          {\small Technik, Wirtschaft und Gestaltung}

          \textsf{\Large Fakultät Informatik} \\
          Studiengang \studiengang
          \end{center}

  \begin{center}

  \vspace*{1cm}

  \begin{tabular}{p{3cm}p{10cm}}
  Thema: & \textbf{\large \thema} \\[2ex]
  Masterkandidat: & \autor, \autorStrasse, \autorPLZ ~ \autorOrt \\[2ex]
  Firma: & \firma \\[2ex]
  1. Prüfer: & \prueferA \\
  2. Prüfer: & \prueferB \\[4ex]
  Schlagworte: & \keywords \\[4ex]
  Ausgabedatum: & \ausgabedatum \\
  \ifdraft Entwurf-Version: \else Abgabedatum: \fi & \abgabedatum \\
  \end{tabular}
  \end{center}

\intentionallyBlankPage

% Abstract

  \newpage
  %\setcounter{page}{3}  % Count the title pages (Conform with DIN 1422-4 Section 4.1)
  \sectionchapter{Kurzreferat}
  \abstract
  \sectionchapter{Abstract}
  \abstractEngl

% Affidavit

  \newpage
  \sectionchapter{Ehrenwörtliche Erklärung}

  Hiermit erkläre ich
  \textit{\autor, geboren am \autorGeburtsdatum~in \autorGeburtsort}, dass ich\\

  \begin{tabular}{lp{12cm}}
  (1) & meine Masterarbeit mit dem Titel \\[1em]
  & \textbf{\thema} \\[1em]
  & beim \firma\ unter Anleitung von \prueferA\ selbständig und ohne fremde Hilfe angefertigt und keine anderen als die angeführten Hilfen benutzt habe;\\[1em]
  (2) & die Übernahme wörtlicher Zitate, von Tabellen, Zeichnungen, Bildern und
  Programmen aus der Literatur oder anderen Quellen (Internet) sowie die Verwendung
  der Gedanken anderer Autoren an den entsprechenden Stellen innerhalb der Arbeit
  gekennzeichnet habe.\\
  \end{tabular}


  \noindent
  Ich bin mir bewusst, dass eine falsche Erklärung rechtliche Folgen haben wird.\\

  \vspace*{0.5cm}

  \noindent
  Konstanz, \abgabedatum \hfill \begin{tabular}{c} \\ \\ \rule{5cm}{1pt} \\ (Unterschrift)\end{tabular}

  \intentionallyBlankPage

}
  % Choose one Titlepage (must provide \typesetTitle)

\newcommand{\fig}[2]{
  \begin{figure}[h!]
    \centering
    \advance\leftskip-2.5cm
    \fbox{\includegraphics[width=1.45\textwidth]{figures/#1.svg.pdf}}
    \caption{#2}\label{fig.#1}
  \end{figure}
}

\newcommand{\titlepageAreal}{
  \pagenumbering{roman}
  \thispagestyle{empty}
  \typesetTitle
  \newpage
  \clearpage
}
\newcommand{\tableOfContentsAreal}{
  \begin{singlespace}
  \tableofcontents
  \end{singlespace}
}
\newcommand{\bibliographyAreal}{
  \newpage

  \ifundef{\chapter}{
    \addcontentsline{toc}{section}{Literaturverzeichnis}
    \typeout{Bibliography now uses section heading.}
  }{
    \addcontentsline{toc}{chapter}{Literaturverzeichnis}
    \typeout{Bibliography now uses chapter heading.}
  }

  % gerabbrv == [1], [2], ...
  % apalike == [Autor, 1987], ...
  % alphadin == [Au87], ...
  % natdin == DIN 1505 == (Autor 1987), ...
  \bibliographystyle{natdin}
  \bibliography{Bibliography}
}
\newcommand{\sectionchapter}[1]{
  \ifundef{\chapter}{
    \section*{#1}
    \addcontentsline{toc}{section}{#1}
  }{
    \chapter*{#1}
    \addcontentsline{toc}{chapter}{#1}
  }
}

