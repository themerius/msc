\setboolean{draft}{true}

\newcommand{\autor}{Sven Hodapp}
\newcommand{\documenttype}{Masterarbeit}
\newcommand{\thema}{Entwicklung eines browserbasierten Dokumenteditors für (wiss.) Texte
mit domänenspezifischen Inhalten}
\newcommand{\subtitle}{
  Lorem ipsum dolor sit amet, consetetur sadipscing elitr
}
\newcommand{\institute}{
  \begin{tabular}{r l}
    Institution: & Hochschule Konstanz \\
    Bereich: & Fakultät Informatik \\
    Betreuer: & Prof. Dr. John Doe \\
    Version: & \today
  \end{tabular}
}
\newcommand{\keywords}{expose}
\renewcommand{\abstract}{Maximal 1200 Anschläge, alles in einem Abschnitt. (Nach DIN 1422-1).}

\newcommand{\ausgabedatum}{1. März 2014}
\newcommand{\abgabedatum}{\today}
\newcommand{\autorStrasse}{Hohentwielstraße 2}
\newcommand{\autorPLZ}{78247}
\newcommand{\autorOrt}{Hilzingen}
\newcommand{\autorGeburtsort}{Singen am Hohentwiel}
\newcommand{\autorGeburtsdatum}{16. September 1987}
\newcommand{\prueferA}{Prof. Dr. Marko Boger}
\newcommand{\prueferB}{Dr. Marc Zimmermann}
\newcommand{\firma}{Fraunhofer-Institut für Algorithmen und Wissenschaftliches Rechnen SCAI}
\newcommand{\studiengang}{Master of Science Informatik}

\newcommand{\typesetTitle}{

% Cover

  \begin{titlepage}

  \vspace*{-3.5cm}

  \begin{flushleft}
  \hspace*{-1cm} \includegraphics[width=15.7cm]{Titlepages/htwg-logo}
  \end{flushleft}

  \vspace{2.5cm}

  \begin{center}
    \huge{
      \textbf{\thema} \\[4cm]
    }
    \Large{
      \textbf{\autor}} \\[4cm]
    \large{
      \textbf{Konstanz, \abgabedatum \ifdraft ~(ENTWURF) \fi} \\[2cm]
    }
    
    \Huge{
      \textbf{{\sf MASTERARBEIT}}
    }
  \end{center}

  \end{titlepage}

\setcounter{page}{2}
\intentionallyBlankPage

% Title

  \thispagestyle{empty}
  \setlength{\parskip}{0.5cm}
          \begin{center}
          \textbf{\huge MASTERARBEIT}

          \textbf{zur Erlangung des akademischen Grades}

          \textbf{\Large Master of Science (M. Sc.)}

          \textbf{an der}

          \textsf{\huge Hochschule Konstanz}\\
          {\small Technik, Wirtschaft und Gestaltung}

          \textsf{\Large Fakultät Informatik} \\
          Studiengang \studiengang
          \end{center}

  \begin{center}

  \vspace*{1cm}

  \begin{tabular}{p{3cm}p{10cm}}
  Thema: & \textbf{\large \thema} \\[2ex]
  Masterkandidat: & \autor, \autorStrasse, \autorPLZ ~ \autorOrt \\[2ex]
  Firma: & \firma \\[2ex]
  1. Prüfer: & \prueferA \\
  2. Prüfer: & \prueferB \\[4ex]
  Schlagworte: & \keywords \\[4ex]
  Ausgabedatum: & \ausgabedatum \\
  \ifdraft Entwurf-Version: \else Abgabedatum: \fi & \abgabedatum \\
  \end{tabular}
  \end{center}

\intentionallyBlankPage

% Abstract

  \newpage
  %\setcounter{page}{3}  % Count the title pages (Conform with DIN 1422-4 Section 4.1)
  \sectionchapter{Kurzreferat}
  \abstract
  \sectionchapter{Abstract}
  \abstractEngl

% Affidavit

  \newpage
  \sectionchapter{Ehrenwörtliche Erklärung}

  Hiermit erkläre ich
  \textit{\autor, geboren am \autorGeburtsdatum~in \autorGeburtsort}, dass ich\\

  \begin{tabular}{lp{12cm}}
  (1) & meine Masterarbeit mit dem Titel \\[1em]
  & \textbf{\thema} \\[1em]
  & beim \firma\ unter Anleitung von \prueferA\ selbständig und ohne fremde Hilfe angefertigt und keine anderen als die angeführten Hilfen benutzt habe;\\[1em]
  (2) & die Übernahme wörtlicher Zitate, von Tabellen, Zeichnungen, Bildern und
  Programmen aus der Literatur oder anderen Quellen (Internet) sowie die Verwendung
  der Gedanken anderer Autoren an den entsprechenden Stellen innerhalb der Arbeit
  gekennzeichnet habe.\\
  \end{tabular}


  \noindent
  Ich bin mir bewusst, dass eine falsche Erklärung rechtliche Folgen haben wird.\\

  \vspace*{0.5cm}

  \noindent
  Konstanz, \abgabedatum \hfill \begin{tabular}{c} \\ \\ \rule{5cm}{1pt} \\ (Unterschrift)\end{tabular}

  \intentionallyBlankPage

}
  % Choose one Titlepage (must provide \typesetTitle)

\newcommand{\fig}[2]{
  \begin{figure}[h!]
    \centering
    \advance\leftskip-2.5cm
    \fbox{\includegraphics[width=1.45\textwidth]{figures/#1.svg.pdf}}
    \caption{#2}\label{fig.#1}
  \end{figure}
}

\newcommand{\titlepageAreal}{
  \pagenumbering{roman}
  \thispagestyle{empty}
  \typesetTitle
  \newpage
  \clearpage
}
\newcommand{\tableOfContentsAreal}{
  \begin{singlespace}
  \tableofcontents
  \end{singlespace}
  \onehalfspacing
  \newpage
  \pagenumbering{arabic}
}
\newcommand{\bibliographyAreal}{
  \newpage

  \ifundef{\chapter}{
    \addcontentsline{toc}{section}{Literaturverzeichnis}
    \typeout{Bibliography now uses section heading.}
  }{
    \addcontentsline{toc}{chapter}{Literaturverzeichnis}
    \typeout{Bibliography now uses chapter heading.}
  }

  % gerabbrv == [1], [2], ...
  % apalike == [Autor, 1987], ...
  % alphadin == [Au87], ...
  % natdin == DIN 1505 == (Autor 1987), ...
  \bibliographystyle{natdin}
  \bibliography{Bibliography}
}
\newcommand{\sectionchapter}[1]{
  \ifundef{\chapter}{
    \section*{#1}
    \addcontentsline{toc}{section}{#1}
  }{
    \chapter*{#1}
    \addcontentsline{toc}{chapter}{#1}
  }
}

